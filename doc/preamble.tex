% ---------- Шрифты ----------------
\usepackage{fontspec}                
\setmainfont{Times New Roman}

\usepackage{unicode-math}
\setmathfont{latinmodern-math.otf}

\newfontfamily\cyrillicfonttt[Script=Cyrillic]{Times New Roman}

% ---------- Языки -----------------
\usepackage{polyglossia}             
\setmainlanguage{russian}            
\setotherlanguage{english}           

% ---------- Разметка страницы --------------
\usepackage{geometry}
\geometry{top=2cm, bottom=2cm, left=3cm, right=1.5cm}

\setlength{\parindent}{1.5cm}       
\usepackage{indentfirst}

% ---------- Интервалы -------------
\usepackage{setspace} 
\onehalfspacing 

% ---------- Таблицы и графика -----------
\usepackage{booktabs}  % красивые таблицы
\usepackage{tabularx}  
\usepackage{array}     
% \usepackage[draft]{graphicx}  % вставка картинок
\usepackage{graphicx}  % вставка картинок
\usepackage{rotating}  % боковые фигуры
\usepackage{floatrow}  
\usepackage{caption}

\floatsetup[table]{capposition=top}

\setlength{\floatsep}{6pt}
\setlength{\textfloatsep}{12pt}
\setlength{\intextsep}{12pt}

\addto\captionsrussian{%
  \renewcommand{\figurename}{Рисунок}
}

\captionsetup{
  font=small
}

% ---------- Списки и форматирование ----------
\usepackage{enumitem}
\usepackage{titlesec}
\usepackage{needspace}
\setlist[enumerate]{
  leftmargin=1.5cm,
  topsep=0em,
  itemsep=0em,
  % labelsep=0em
}
\setlist[itemize]{
  leftmargin=1.5cm,
  topsep=0em,
  itemsep=0em,
  % labelsep=0em,
}


% ---------- Математика ----------
\usepackage{amsmath}        
\numberwithin{equation}{section}

% ---------- Собственные команды -------------
\newcommand{\deriv}[2]{\frac{\partial #1}{\partial #2}}
\newcommand{\dd}[2]{\frac{d #1}{d #2}}
\newcommand{\R}{\mathbb{R}}

\newcommand{\rotatefigure}[3]{%
\begin{sidewaysfigure}
\centering
\includegraphics[width=#1\textheight]{#2}
\caption{#3}
\end{sidewaysfigure}
}

\newcolumntype{P}[1]{>{\centering\arraybackslash}p{#1}}

% ---------- Оглавление (TOC) ----------
\usepackage{tocloft}

\addto\captionsrussian{\renewcommand{\contentsname}{СОДЕРЖАНИЕ}}
\renewcommand{\cfttoctitlefont}{\hspace*{\fill}\normalfont\normalsize\bfseries}
\renewcommand{\cftaftertoctitle}{\hspace*{\fill}\par}
\renewcommand{\cftsecfont}{\normalfont\normalsize}
\renewcommand{\cftsubsecfont}{\normalfont\normalsize}
\renewcommand{\cftsecpagefont}{\normalfont\normalsize}
\renewcommand{\cftsubsecpagefont}{\normalfont\normalsize}
\renewcommand{\cftsecleader}{\cftdotfill{\cftdotsep}}

\setlength{\cftbeforesecskip}{0pt}
\setlength{\cftbeforesubsecskip}{0pt}
\setlength{\cftbeforesubsubsecskip}{0pt}
\setlength{\cftsecnumwidth}{1.5em}
\setlength{\cftsubsecnumwidth}{2em}

% ---------- Нумерация секций и формул ----------
\renewcommand{\thesection}{\arabic{section}.}
\renewcommand{\thesubsection}{\arabic{section}.\arabic{subsection}.}
\renewcommand{\theequation}{\thesection\arabic{equation}.}

\titleformat{\section}
  {\normalfont\normalsize\bfseries\centering}
  {\thesection}
  {0.5em}
  {}

\titleformat{\subsection}
  {\normalfont\normalsize\centering}
  {\thesubsection}
  {0.5em}
  {}

\titlespacing*{\section}{0pt}{0pt}{0pt}
\titlespacing*{\subsection}{0pt}{0pt}{\baselineskip}

% ---------- Настройки для формул ----------
\makeatletter
\g@addto@macro\normalsize{
  \setlength{\abovedisplayskip}{12pt}%
  \setlength{\belowdisplayskip}{12pt}%
  \setlength{\abovedisplayshortskip}{0pt}%
  \setlength{\belowdisplayshortskip}{14pt}%
}
\makeatother
% ------------------ TikZ -----------------
\usepackage{tikz}
\usetikzlibrary{shapes.geometric, arrows}

% --------------- Ссылки ------------------
\usepackage{hyperref}
\usepackage{url}

