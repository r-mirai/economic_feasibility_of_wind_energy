\section{\centering ВВЕДЕНИЕ}

В условиях глобального роста энергопотребления и одновременного обострения экологических и ресурсных ограничений традиционной энергетики особую актуальность приобретает развитие возобновляемых источников энергии. Среди них ветроэнергетика занимает одно из ведущих мест благодаря технологической зрелости, масштабируемости и отсутствию прямых выбросов загрязняющих веществ в процессе генерации электроэнергии.

Современные ветроэнергетические установки (ВЭУ) представляют собой сложные инженерные системы, эффективность и экономическая целесообразность которых в значительной степени зависят от природно-климатических условий площадки размещения, параметров ветрового режима, а также от корректного выбора и расчёта основных компонентов установки. Неправильная оценка этих факторов может привести к существенному снижению энергетической отдачи и росту приведённой стоимости электроэнергии.

Особую важность в настоящее время приобретает использование долгосрочных метеорологических данных при обосновании проектов ветроэнергетики. Анализ многолетних рядов скорости и распределения ветра позволяет повысить достоверность оценки потенциальной выработки электроэнергии, а также выполнить более корректный экономический анализ, учитывающий изменчивость климатических условий во времени. В этой связи применение статистических данных за продолжительный период — с 2001 по 2024 годы — позволяет получить репрезентативные результаты и снизить неопределённость расчётов.

Целью данной курсовой работы является расчёт параметров ветроэнергетических установок и оценка их технико-экономической эффективности на основе многолетних данных ветрового режима для трёх различных городов. В рамках работы выполняется анализ исходных метеорологических данных, расчёт основных энергетических характеристик ВЭУ, подбор ключевых компонентов, а также экономическая оценка проекта с использованием графического анализа полученных результатов.

\needspace{4\baselineskip}
Для достижения поставленной цели в работе решаются следующие задачи:

\begin{itemize}[topsep=0pt, itemsep=0pt]
    \item анализ ветровых условий исследуемых регионов на основе данных за 2001–2024 годы
    \item расчёт энергетических параметров ветроэнергетических установок с учётом высоты башни и номинальной мощности
    \item оценка выработки электроэнергии и коэффициента использования установленной мощности
    \item проведение экономического расчёта, включающего капитальные и эксплуатационные затраты
    \item визуализация результатов в виде графиков и сравнительный анализ для трёх городов.
\end{itemize}

Объектом исследования является ветроэнергетическая установка как элемент системы возобновляемой энергетики.
Предметом исследования являются параметры компонентов ВЭУ и их влияние на энергетическую и экономическую эффективность при различных ветровых условиях.

Практическая значимость работы заключается в возможности использования полученных результатов при предварительном технико-экономическом обосновании проектов ветроэнергетики, а также в учебных и исследовательских целях при анализе эффективности возобновляемых источников энергии.