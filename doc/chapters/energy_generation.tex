\section{РАСЧЕТ ВЫРАБОТКИ ЭНЕРГИИ}
\subsection{Интерполяция кривой вырабатываемой мощности ВЭУ}

В работе были использованы данные о мощности ветроэнергетической установки,
заданные в виде значений в десяти дискретных точках в зависимости от скорости ветра.
Для корректного моделирования работы установки требуется более детализированная зависимость
в диапазоне скоростей от 0 до 30~м/с с шагом не более 0.5~м/с.
В связи с этим была выполнена интерполяция исходных данных.

Полученная кривая мощности может быть разделена на четыре характерные области:

\begin{enumerate}[topsep=0pt, itemsep=0pt]
    \item \textbf{Область роста} $v \in \left[ v_{\text{start}},\, v_{\text{nominal}} \right]$.
    В данном диапазоне ветрогенератор постепенно выходит на номинальный режим работы,
    при этом вырабатываемая мощность монотонно возрастает с увеличением скорости ветра.

    \item \textbf{Плато} $v \in \left[ v_{\text{nominal}},\, v_{\text{derate-start}} + 3 \right]$.
    В этой области мощность установки остаётся практически постоянной
    и составляет около $1.2\,P_{\text{nominal}}$.

    \item \textbf{Область снижения} $v \in \left[ v_{\text{derate-start}} + 3,\, v_{\text{cut-out}} \right]$.
    Уменьшение вырабатываемой мощности в данном диапазоне обусловлено
    дополнительными потерями энергии на торможение ротора,
    а также возрастанием уровня турбулентности воздушного потока.

    \item \textbf{Режим cut-out} $v > v_{\text{cut-out}}$.
    При превышении предельной скорости ветра происходит полная остановка установки
    с целью предотвращения её механических повреждений.
\end{enumerate}

\pagebreak

Ниже приведено выражение, используемое для интерполяции кривой мощности.

\begin{equation*}
p(v) =
\begin{cases}

0 & v \le v_{\text{start}} \\[12pt]

\dfrac{1.2\,p_{\text{nominal}}}
{1 + e^{-a\left(v - b\right)}} 
& v_{\text{start}} < v \le v_{\text{derate start}} \\[12pt]

\dfrac{1.2\,p_{\text{nominal}}}
{1 + e^{-a\left(v - b\right)}} - 
\dfrac{p(v_{\text{derate start}})}{v_{\text{derate start}}^3}\left(v - v_{\text{derate start}}\right)^3
& v_{\text{derate start}} < v \le v_{\text{cut-out}} \\[12pt]

p(v_{\text{cut-out}})\,
e^{-\left(v - v_{\text{cut-out}}\right)} 
& v_{\text{cut-out}} < v \le v_{\text{cut-out}} + 1.5 \\[12pt]

0 &  v_{\text{cut-out}} + 1.5 < v 

\end{cases}
\end{equation*}
\hspace{0.5em}

Здесь коэффициенты $a$ и $b$ подбираются таким образом,
чтобы полученная зависимость наиболее точно аппроксимировала имеющиеся исходные данные.
Остальные параметры были зафиксированы и приняты равными типичным значениям,
встречающимся в литературе.
Величина $p_{\text{nominal}}$ оставлена в виде параметра
для последующего анализа.
Значения используемых констант приведены ниже.

\begin{itemize}[topsep=0pt, itemsep=0pt]
    \item $v_{\text{start}} = 3~\text{м/с}$
    \item $v_{\text{nominal}} = 10.5~\text{м/с}$
    \item $v_{\text{derate start}} = 15~\text{м/с}$
    \item $v_{\text{cut-out}} = 25~\text{м/с}$
    \item $p_{\text{start}} = 0.025\,p_{\text{nominal}}$
\end{itemize}

\pagebreak

Далее приведены графики интерполяции,
а также их сравнение с исходными экспериментальными данными.

\begin{figure}[!htbp]
    \centering
    \caption{Интерполяция данных генерации ВЭУ-3}
    \includegraphics[width=0.95\textwidth]{figures/3_Интерполяция мощности ВЭУ-3.png}
\end{figure}

\begin{figure}[!htbp]
    \centering
    \caption{Интерполяция данных генерации ВЭУ-5}
    \includegraphics[width=0.95\textwidth]{figures/3_Интерполяция мощности ВЭУ-5.png}
\end{figure}

\pagebreak
\subsection{Анализ данных мощности}

