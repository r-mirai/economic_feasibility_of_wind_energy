\section{РАСЧЕТ ВЫРАБОТКИ ЭНЕРГИИ}
\subsection{Аппроксимация кривой вырабатываемой мощности ВЭУ}

В работе были использованы данные о мощности ветроэнергетической установки,
заданные в виде значений в десяти дискретных точках в зависимости от скорости ветра.
Для корректного моделирования работы установки требуется более детализированная зависимость
в диапазоне скоростей от 0 до 30~м/с с шагом не более 0.5~м/с.
В связи с этим была выполнена аппроксимировала исходных данных.

Полученная кривая мощности может быть разделена на четыре характерные области:

\begin{enumerate}[topsep=0pt, itemsep=0pt]
    \item \textbf{Область роста} $v \in \left[ v_{\text{start}},\, v_{\text{nominal}} \right]$.
    В данном диапазоне ветрогенератор постепенно выходит на номинальный режим работы,
    при этом вырабатываемая мощность монотонно возрастает с увеличением скорости ветра.

    \item \textbf{Плато} $v \in \left[ v_{\text{nominal}},\, v_{\text{derate-start}} + 3 \right]$.
    В этой области мощность установки остаётся практически постоянной
    и составляет около $1.2\,P_{\text{nominal}}$.

    \item \textbf{Область снижения} $v \in \left[ v_{\text{derate-start}} + 3,\, v_{\text{cut-out}} \right]$.
    Уменьшение вырабатываемой мощности в данном диапазоне обусловлено
    дополнительными потерями энергии на торможение ротора,
    а также возрастанием уровня турбулентности воздушного потока.

    \item \textbf{Режим cut-out} $v > v_{\text{cut-out}}$.
    При превышении предельной скорости ветра происходит полная остановка установки
    с целью предотвращения её механических повреждений.
\end{enumerate}

\pagebreak

Ниже приведено выражение, используемое для аппроксимации кривой мощности.

\begin{equation*}
p(v) =
\begin{cases}

0 & v \le v_{\text{start}} \\[12pt]

\dfrac{1.2\,p_{\text{nominal}}}
{1 + e^{-a\left(v - b\right)}} 
& v_{\text{start}} < v \le v_{\text{derate start}} \\[12pt]

\dfrac{1.2\,p_{\text{nominal}}}
{1 + e^{-a\left(v - b\right)}} - 
\dfrac{p(v_{\text{derate start}})}{v_{\text{derate start}}^3}\left(v - v_{\text{derate start}}\right)^3
& v_{\text{derate start}} < v \le v_{\text{cut-out}} \\[12pt]

p(v_{\text{cut-out}})\,
e^{-\left(v - v_{\text{cut-out}}\right)} 
& v_{\text{cut-out}} < v \le v_{\text{cut-out}} + 1.5 \\[12pt]

0 &  v_{\text{cut-out}} + 1.5 < v 

\end{cases}
\end{equation*}

Здесь коэффициенты $a$ и $b$ подбираются таким образом,
чтобы полученная зависимость наиболее точно аппроксимировала имеющиеся исходные данные.
Остальные параметры были зафиксированы и приняты равными типичным значениям,
встречающимся в литературе.
Величина $p_{\text{nominal}}$ оставлена в виде параметра
для последующего анализа.
Значения используемых констант приведены ниже.

\begin{itemize}[topsep=0pt, itemsep=0pt]
    \item $v_{\text{start}} = 3~\text{м/с}$
    \item $v_{\text{nominal}} = 10.5~\text{м/с}$
    \item $v_{\text{derate start}} = 15~\text{м/с}$
    \item $v_{\text{cut-out}} = 25~\text{м/с}$
    \item $p_{\text{start}} = 0.025\,p_{\text{nominal}}$
\end{itemize}

\pagebreak

Ниже приведены графики аппроксимации 
и их сравнение с исходными экспериментальными данными.

\begin{figure}[!htbp]
    \centering
    \caption{Аппроксимация данных генерации ВЭУ-3}
    \includegraphics[width=0.95\textwidth]{figures/3_Аппроксимация мощности ВЭУ-3.png}
\end{figure}

\begin{figure}[!htbp]
    \centering
    \caption{Аппроксимация данных генерации ВЭУ-5}
    \includegraphics[width=0.95\textwidth]{figures/3_Аппроксимация мощности ВЭУ-5.png}
\end{figure}

\pagebreak
\subsection{Анализ данных мощности}

Имея данные скорости ветра и таблицу преобразования 
их в вырабатываемую мощность, 
можно получать значения мощности за любой период времени.

Приведем несколько графиков для более детального ознакомления с данными

\begin{figure}[!htbp]
    \centering
    \caption{Империческое распределение ВЭУ-1 в городе Казань}
    \includegraphics[width=0.9\textwidth]{figures/3_Эмпирическое распределение мощности ВЭУ-1 в городе Казань.png}
\end{figure}

\begin{figure}[!htbp]
    \centering
    \caption{Империческое распределение ВЭУ-1 в городе Набережные Челны}
    \includegraphics[width=0.9\textwidth]{figures/3_Эмпирическое распределение мощности ВЭУ-1 в городе Набережные Челны.png}
\end{figure}

\pagebreak
\begin{figure}[!htbp]
    \centering
    \caption{Империческое распределение ВЭУ-1 в городе Таганрог}
    \includegraphics[width=0.9\textwidth]{figures/3_Эмпирическое распределение мощности ВЭУ-1 в городе Таганрог.png}
\end{figure}

Из этих графиков можно определить процентили для трёх городов. 
Например, медиана (50-й процентиль) составляет примерно 
50, 70 и 100 Вт для Казани, Набережных Челнов и Таганрога соответственно 
при высоте 10 метров. 
При той же высоте около 30\% данных для Набережных Челнов и Казани равны нулевой мощности, 
в то время как в Таганроге этот показатель составляет 20\%. 
Теперь рассмотрим средние значения для каждого года.

\begin{figure}[!htbp]
    \centering
    \caption{Средняя мощность генерации ВЭУ-1 в каждом году в городе Казань}
    \includegraphics[width=0.9\textwidth]{figures/3_Средняя мощность генерации ВЭУ-1 в каждом году в городе Казань.png}
\end{figure}

\pagebreak

\begin{figure}[!htbp]
    \centering
    \caption{Средняя мощность генерации ВЭУ-1 в каждом году в городе Набережные Челны}
    \includegraphics[width=0.9\textwidth]{figures/3_Средняя мощность генерации ВЭУ-1 в каждом году в городе Набережные Челны.png}
\end{figure}

\begin{figure}[!htbp]
    \centering
    \caption{Средняя мощность генерации ВЭУ-1 в каждом году в городе Таганрог}
    \includegraphics[width=0.9\textwidth]{figures/3_Средняя мощность генерации ВЭУ-1 в каждом году в городе Таганрог.png}
\end{figure}

На графиках отсутствуют аномально высокие или низкие значения. 
Исключение составляет Таганрог в 2014 году, 
где наблюдается пик, превышающий среднее значение на 30\% при высоте 30~м. 
Этот год действительно был штормовым для Таганрога; 
в сентябре даже зафиксирован 12-балльный ураган.