\section{ОПРЕДЕЛЕНИЕ ХАРАКТЕРИСТИК ВЕТРА}
\subsection{Получение и хранение данных}

Для анализа использовались данные о скоростях ветра, предоставляемые NASA.
В открытом доступе представлены значения скорости ветра только на высотах 10 и 50 метров.
Для определения скорости ветра на промежуточных высотах применяется эмпирическая аппроксимация с последующим усреднением значений,
что описывается следующей формулой:

\begin{equation}
v_h = \frac{1}{2}
\left(
v_{10}
\left(\frac{h}{10}\right)^{\alpha}
+
v_{50}
\left(\frac{h}{50}\right)^{\alpha}
\right)
\label{eq:wind_profile}
\end{equation}

Здесь $\alpha$ — коэффициент шероховатости, подбираемый отдельно для каждого города с целью аппроксимации профиля скорости ветра
между высотами 10 и 50 м.
Минимальное значение коэффициента $\alpha$ наблюдается для города Ялта и составляет 0.09,
максимальное — для города Нелидово и равно 0.43.

Вычисления выполнялись на машине с объёмом свободной оперативной памяти около 2 Гб,
что накладывало ограничения на хранение и обработку данных.
Почасовые данные по всем городам России за период 2001--2024 годов занимают около 3.3 Гб,
в связи с чем для работы с ними использовались библиотеки \texttt{zarr} и \texttt{dask}.
Данные библиотеки позволяют хранить массивы данных на жёстком диске,
загружать в память только необходимые фрагменты и выполнять вычисления без полной загрузки набора данных,
что делает возможной работу с объёмами данных порядка 200--300 Гбц.

\pagebreak
\subsection{Анализ полученных данных}
\begin{table}[h]
\centering
\caption{Медианная скорость ветра за 2001-2024 года}
\begin{tabular}{l P{1.2cm} P{1.2cm} P{1.2cm}}
\toprule
Высота, м & 10 & 20 & 30 \\
Город &  \multicolumn{3}{c}{Скорость ветра, м/с}  \\
\midrule
Северо-Курильск & 6.2 & 6.8 & 7.1 \\
Пионерский & 6.1 & 6.7 & 7.0 \\
Светлогорск & 6.1 & 6.7 & 7.0 \\
Курильск & 6.0 & 6.6 & 6.9 \\
Островной & 5.7 & 6.5 & 6.9 \\
Зеленоградск & 5.4 & 6.1 & 6.5 \\
Керчь & 5.5 & 6.0 & 6.3 \\
Оха & 5.3 & 6.0 & 6.4 \\
Щелкино & 5.3 & 5.8 & 6.1 \\
Феодосия & 5.3 & 5.8 & 6.0 \\
\dots & \dots & \dots & \dots \\
Урус-Мартан & 1.9 & 2.3 & 2.5 \\
Туран & 1.9 & 2.2 & 2.5 \\
Чадан & 1.9 & 2.2 & 2.4 \\
Беслан & 1.8 & 2.0 & 2.1 \\
Дигора & 1.8 & 2.0 & 2.1 \\
Алагир & 1.8 & 2.0 & 2.1 \\
Ардон & 1.8 & 2.0 & 2.1 \\
Владикавказ & 1.8 & 2.0 & 2.1 \\
Назрань & 1.6 & 2.0 & 2.1 \\
Магас & 1.6 & 2.0 & 2.1 \\
\bottomrule
\end{tabular}
\end{table}

На основании полученных данных были определены наиболее ветреные города России.
Для дальнейшего анализа были выбраны три города: Казань, Набережные Челны и Таганрог,
занимающие 182, 69 и 18 места соответственно из 1097.
Далее рассматриваются профили скорости ветра в зависимости от времени суток и месяца.

\rotatefigure{0.95}{figures/1_Средняя скорость ветра в городе Казань.png}{Средняя скорость ветра в городе Казань}
\rotatefigure{0.95}{figures/1_Средняя скорость ветра в городе Набережные Челны.png}{Средняя скорость ветра в городе Набережные Челны}
\rotatefigure{0.95}{figures/1_Средняя скорость ветра в городе Таганрог.png}{Средняя скорость ветра в городе Таганрог}

\pagebreak
Анализ суточных профилей показывает, что скорость ветра существенно зависит от времени суток.
Вблизи полудня средняя скорость ветра, как правило, на 1 м/с выше, чем в вечерние или ночные часы.
Данная особенность указывает на то, что для потребителей с различным профилем энергопотребления
окупаемость ветроустановки может существенно различаться.
Также наблюдается выраженная сезонная зависимость, которая наглядно представлена на следующих графиках.

\begin{figure}[!htbp]
\centering
\includegraphics[width=0.90\textwidth]{figures/1_Среднее кол-во часов с ветром больше 5 м:с в городе Казань.png}
\caption{Среднее количество часов с ветром более 5 м/с в городе Казань}
\end{figure}

\begin{figure}[!htbp]
\centering
\includegraphics[width=0.90\textwidth]{figures/1_Среднее кол-во часов с ветром больше 5 м:с в городе Набережные Челны.png}
\caption{Среднее количество часов с ветром более 5 м/с в городе Набережные Челны}
\end{figure}

\pagebreak
\begin{figure}[!htbp]
\centering
\includegraphics[width=0.90\textwidth]{figures/1_Среднее кол-во часов с ветром больше 5 м:с в городе Таганрог.png}
\caption{Среднее количество часов с ветром более 5 м/с в городе Таганрог}
\end{figure}

Сезонная динамика показывает заметное снижение скорости ветра в летний период.
Для города Набережные Челны среднее количество часов,
в течение которых генератор способен выдавать значительную мощность,
в отдельные месяцы снижается до значений менее пяти часов.

С учётом выявленных особенностей для корректного определения окупаемости
недостаточно использовать только усреднённые значения.
Необходимо учитывать конкретный профиль энергопотребления.
В идеальном случае требуется зависимость энергопотребления для каждого часа года,
однако при отсутствии детализированных данных используется средняя эмпирически построенная суточная кривая,
распространённая на все дни года.
