\section{ЗАКЛЮЧЕНИЕ}
\vspace{1em}

В рамках данной работы был проведён экономический анализ установки ветроэнергетической установки (ВЭУ) для трёх городов: Казани, Набережных Челнов и Таганрога. 

Были выполнены следующие этапы работы:
\begin{itemize}
    \item Получение и анализ данных скорости ветра в этих городах за 24 года.
    \item Формирование данных потребления электроэнергии с использованием вероятностного метода.
    \item Аппроксимация кривой выработки энергии ветрогенератором.
    \item Анализ данных выработки электроэнергии.
    \item Моделирование работы всей системы, что позволило получить оценки потребленной и избыточной электроэнергии.
    \item Сбор и аппроксимация цен на оборудование (генератор, контроллер, инвертор, мачту и аккумуляторную батарею), исходя из данных интернет-магазинов, преимущественно китайской компании Zonhan.
    \item Разработка функции расчёта полной стоимости ВЭУ с учётом транспортировки, монтажа, обслуживания, разрешений и непредвиденных расходов.
    \item Расчёт окупаемости ВЭУ для разных мощностей и высот башни с учётом розничных тарифов на электроэнергию.
\end{itemize}

Результаты показали, что во всех трёх городах установка ВЭУ является экономически обоснованной. Например, 12~кВт установка в Казани окупается примерно за 20 лет. 

\pagebreak
В работе использовались следующие допущения, которые могут влиять на точность результатов:

\begin{itemize}
    \item \textbf{Данные ветра NASA:} использовались данные NASA POWER, которые представляют собой результаты глобального реанализа и моделей атмосферы, объединяющих спутниковые наблюдения и данные метеостанций. Эти данные усреднены по крупной пространственной ячейке и не учитывают локальные особенности рельефа, застройку, турбулентность и микроклиматические эффекты. В зависимости от конкретных условий местности погрешность оценки скорости ветра может достигать 10–20\%, что особенно важно учитывать при расчёте выработки энергии, чувствительной к кубу скорости ветра.
    \item \textbf{Энергопотребление формируется случайным образом:} профиль энергопотребления формируется без учёта корреляции включения между некоторыми приборами и без учёта сезонных и суточных колебаний, что может влиять на точность моделирования кратковременных пиков нагрузки.
    \item \textbf{Аппроксимация кривой выработки энергии ветрогенератором:} реальные характеристики разных моделей ВЭУ могут отличаться, что вносит дополнительную неопределённость в расчётах.
    \item \textbf{Аппроксимация стоимости компонентов:} для каждого компонента использовалось от 10 до 20 ценовых точек из открытых источников, что могло повлиять на точность апроксимации.
    \item \textbf{Тариф на электроэнергию:} для упрощения расчётов принят фиксированный тариф, без учёта сезонных, часовых изменений и индивидуальных особенностей, что также может влиять на точность оценки окупаемости.
    \item \textbf{Прочие допущения:} не учитывались возможные изменения цен на оборудование в будущем, субсидии, а также изменения политики выкупа избыточной электроэнергии.
\end{itemize}

Несмотря на перечисленные ограничения, проведённый экономический расчёт позволяет сделать обоснованные выводы о целесообразности установки ВЭУ для бытовых потребителей в рассматриваемых городах и может служить основой для дальнейших детализированных исследований. Для значительного увеличения точности рекомендуется для конкретного заказчика фиксировать все неизвестные параметры: измерить его энергопотребление, скорость ветра на месте установки в течение длительного времени, уточнить тариф на электроэнергию и профиль выработки энергии для выбранного класса ветряков.
