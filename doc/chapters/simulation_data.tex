\section{МОДЕЛИРОВАНИЕ РАБОТЫ ВСЕЙ СИСТЕМЫ}
\subsection{Описание модели}
В данной главе проводится моделирование работы всей энергосистемы с целью оценки ключевых показателей: 
остатка энергии после удовлетворения нагрузки, 
общего объёма потреблённой энергии 
и количества часов автономной работы при заданных параметрах модели. 
Схема алгоритма представлена ниже.

\begin{figure}[h!]
\centering
    \begin{tikzpicture}[node distance=4cm, auto]

    % Стили блоков
    \tikzstyle{source} = [rectangle, minimum width=3cm, minimum height=1cm, text centered, draw=black]
    \tikzstyle{battery} = [rectangle, minimum width=3cm, minimum height=1cm, text centered, draw=black]
    \tikzstyle{consumer} = [rectangle, minimum width=3cm, minimum height=1cm, text centered, draw=black]
    \tikzstyle{arrow} = [thick,->,>=stealth]

    % Узлы
    \node (source)[source]{Источник энергии};
    \node (battery)[battery, below of=source]{Батарея};
    \node (consumer)[consumer, right of=battery, xshift=6cm]{Потребитель};

    % Потоки энергии
    \draw[arrow] (source.south) -- (battery.north) node[midway, right] {1. Зарядка батареи};
    \draw[arrow](battery.east) -- (consumer.west) node[midway, above, sloped] {2. Питание из батареи};
    \draw[arrow](source.east) -| (consumer.north) node[pos=0.3, above] {3. Прямое питание};
    \draw[arrow, dashed](source.north) -- ++(0,1) node[midway, right] {4. Излишки энергии};

    \end{tikzpicture}
\caption{Схема передачи энергии между источником, батареей и потребителем}
\end{figure}

Для каждого шага цикла (в данной модели $\Delta t = 1$ час) выполняется следующая последовательность действий:

\begin{enumerate}
    \item Энергия от генератора поступает на заряд батареи.
    \item Затем энергия от батареи направляется потребителю.
    \item Если потребность потребителя не удовлетворена и у генератора осталась энергия, эта остаточная энергия передается потребителю.
    \item Оставшаяся после этого энергия генератора отправляется в сеть.
    \item В конце цикла уровень заряда батареи уменьшается на $\text{battery loss}=0.005$.
\end{enumerate}

Каждая операция передачи энергии (п. 1--4) выполняется с эффективностью $\text{supplied efficiency}=0.95$ 
и пакетами мощности не более $\text{max supplied}=3$ $\text{кВт} \cdot \text{ч}$.

\pagebreak
Важно отметить, что рассмотренная модель является одной из возможных схем подключения.
Например, можно было бы изменить порядок действий: сначала направлять энергию напрямую потребителю, и только потом заряжать аккумулятор.
Такое изменение, вероятно, повлияло бы на итоговые показатели системы на 5--10\% в ту или иную сторону, поэтому для упрощения анализа данный вариант не рассматривался.

\pagebreak
\subsection{Анализ данных симуляции}

После генерации мы получили вышеупомянутые дынные. 
Приведем несколько графиков. 

\begin{figure}[!htbp] 
    \centering
    \includegraphics[width=0.9\textwidth, draft=false]{figures/4_Доля независимых часов на высоте 10 м для города Казань.png}
    \caption{Доля независимых часов на высоте 10 м для города Казань}
\end{figure}

\begin{figure}[!htbp] 
    \centering
    \includegraphics[width=0.9\textwidth, draft=false]{figures/4_Доля независимых часов на высоте 10 м для города Набережные Челны.png}
    \caption{Доля независимых часов на высоте 10 м для города Набережные Челны}
\end{figure}

\pagebreak
\begin{figure}[!htbp] 
    \centering
    \includegraphics[width=0.9\textwidth, draft=false]{figures/4_Доля независимых часов на высоте 10 м для города Таганрог.png}
    \caption{Доля независимых часов на высоте 10 м для города Таганрог}
\end{figure}

Теперь исходя из наших потребностей мы можем выбрать подходящую мощность ветрогенератора
и объем батареи. Но нужно понимать, что от месяца эти числа тоже будут зависеть. 
Приведем несколько графиков для понимания этой зависимости.

\begin{figure}[!htbp] 
    \centering
    \includegraphics[width=0.9\textwidth, draft=false]{figures/4_Доля независимых часов на высоте 20 м для города Казань при объеме батареи 6000 кВт*ч по месяцам.png}
    \caption{Доля независимых часов на высоте 20 м для города Казань при объеме батареи 6000 $\text{кВт}\cdot\text{ч}$ по месяцам}
\end{figure}

\pagebreak
\begin{figure}[!htbp] 
    \centering
    \includegraphics[width=0.9\textwidth, draft=false]{figures/4_Доля независимых часов на высоте 20 м для города Набережные Челны при объеме батареи 6000 кВт*ч по месяцам.png}
    \caption{Доля независимых часов на высоте 20 м для города Набережные Челны при объеме батареи 6000 $\text{кВт}\cdot\text{ч}$ по месяцам}
\end{figure}

\begin{figure}[!htbp] 
    \centering
    \includegraphics[width=0.9\textwidth, draft=false]{figures/4_Доля независимых часов на высоте 20 м для города Таганрог при объеме батареи 6000 кВт*ч по месяцам.png}
    \caption{Доля независимых часов на высоте 20 м для города Таганрог при объеме батареи 6000 $\text{кВт}\cdot\text{ч}$ по месяцам}
\end{figure}

Следует отметить, что учет сезонной зависимости энергопотребления мог бы сгладить колебания, наблюдаемые на приведенных графиках.