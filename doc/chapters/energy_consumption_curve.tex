\section{КРИВАЯ ЭНЕРГОПОТРЕБЛЕНИЯ}
\subsection{Определение кривой энергопотребления}

Кривая энергопотребления в данной работе определяется эмпирическим образом.
Для её построения сутки были разделены на четыре периода: утро, день, вечер и ночь.
Для каждого периода была выполнена оценка используемых электроприборов и продолжительности их работы.
Результаты оценки приведены в таблице ниже.

\begin{table}[h]
\centering
\caption{Энергопотребление приборов}
\begin{tabular}{lr P{1.7cm} P{1.7cm} P{1.7cm} P{1.7cm} P{1.7cm} }
\toprule
\multicolumn{2}{c}{} & \multicolumn{4}{c}{Время начала, ч} & \\
\cmidrule(lr){3-6}
Электроприборы & Вт & Утро 1ч & День 10ч & Вечер 4ч & Ночь 8ч & Сумма, Вт$\cdot$ч \\
\midrule
Телевизор & 300 & 0.5 & 1.0 & 2.0 & 2.0 & 1650.0 \\
Освещение внутри & 300 & 2.5 & 0.2 & 3.0 & 4.0 & 2910.0 \\
Освещение снаружи & 200 & 0.0 & 0.0 & 0.5 & 0.5 & 200.0 \\
Насос & 700 & 0.4 & 0.4 & 0.4 & 0.4 & 1120.0 \\
Строительные приборы & 2000 & 0.2 & 1.0 & 0.2 & 0.2 & 3200.0 \\
Утюг & 2000 & 0.2 & 0.0 & 0.2 & 0.2 & 1200.0 \\
Ноутбук & 180 & 0.2 & 0.2 & 0.2 & 2.0 & 468.0 \\
Компьютер & 650 & 2.0 & 1.0 & 3.0 & 3.0 & 5850.0 \\
Микроволновая печь & 1200 & 0.3 & 0.3 & 0.3 & 0.3 & 1440.0 \\
Фен & 1800 & 0.1 & 0.0 & 0.0 & 0.1 & 360.0 \\
Холодильник & 150 & 1.5 & 2.0 & 1.0 & 1.5 & 900.0 \\
Кухонные приборы & 1000 & 0.0 & 0.0 & 0.5 & 0.5 & 1000.0 \\
Телефон & 45 & 1.0 & 1.0 & 1.0 & 1.0 & 180.0 \\
Стиральная машина & 1200 & 0.2 & 0.0 & 0.0 & 0.0 & 240.0 \\
Другие приборы & 300 & 2.0 & 2.0 & 2.0 & 2.0 & 2400.0 \\
\midrule
\multicolumn{6}{r}{Итого:} & 23118.0 \\ 
\bottomrule
\end{tabular}
\end{table}
\pagebreak

Предполагая равномерное энергопотребление внутри каждого периода суток,
был построен следующий профиль энергопотребления.

\begin{figure}[!htbp] 
\centering
\includegraphics[width=1\textwidth, draft=false]{figures/2_Профиль энергопотребления.png}
\caption{Профиль энергопотребления}
\end{figure}

Сравнение профилей энергопотребления и скорости ветра показывает их диаметральную противоположность.
Данный факт необходимо учитывать при повышении точности оценки окупаемости ветроустановки.
