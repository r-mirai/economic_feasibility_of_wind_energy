\section{\centering ЭКОНОМИЧЕСКИЙ РАСЧЕТ}
\subsection{Определение цены ВЭУ}

Перейдём к экономическому расчёту ветроэнергетической установки (ВЭУ).
На основании результатов моделирования известны объёмы потреблённой электроэнергии,
а также энергии, переданной в электрическую сеть.
Следовательно, уже может быть определена величина сэкономленных денежных средств.
Остаётся оценить полную стоимость создания и эксплуатации ВЭУ.

Для этого были использованы данные из различных интернет-магазинов и открытых
каталогов оборудования. На их основе определены ориентировочные рыночные цены
на основные компоненты ВЭУ: ветроэлектрический генератор, контроллер заряда,
инвертор и аккумуляторные батареи. Для получения аналитических зависимостей
стоимости от номинальной мощности была выполнена аппроксимация собранных данных.

\begin{figure}[!htbp] 
    \centering
    \includegraphics[width=0.9\textwidth, draft=false]{figures/6_Аппроксимация стоимости ветрогенератора.png}
    \caption{Аппроксимация зависимость стоимости ветрогенератора от номинальной мощности}
\end{figure}

\pagebreak
\begin{figure}[!htbp] 
    \centering
    \includegraphics[width=0.9\textwidth, draft=false]{figures/6_Аппроксимация стоимости инвертора.png}
    \caption{Аппроксимация зависимости стоимости инвертора от номинальной мощности}
\end{figure}

\begin{figure}[!htbp] 
    \centering
    \includegraphics[width=0.9\textwidth, draft=false]{figures/6_Аппроксимация стоимости контроллера.png}
    \caption{Аппроксимация зависимости стоимости контроллера от номинальной мощности}
\end{figure}

\pagebreak
Построение функции стоимости мачты оказалось возможным только с учётом двух параметров:
высоты мачты и номинальной мощности ветроустановки. В результате была получена
следующая эмпирическая зависимость:

\begin{equation}
    cost = 3548 \cdot h^{0.932} \cdot power^{0.866}
\end{equation}

Оценка качества аппроксимации была выполнена путём сравнения расчётных значений
со стоимостью мачт, представленных в открытых интернет-источниках.

\begin{figure}[!htbp] 
    \centering
    \includegraphics[width=0.9\textwidth, draft=false]{figures/6_Оценка качества аппроксимации стоимости мачты.png}
    \caption{Оценка качества аппроксимации стоимости мачты}
\end{figure}

Все приведённые зависимости были получены с использованием линейной регрессии
из библиотеки \texttt{sklearn}. Стоимостные данные для ветроэлектрического генератора,
контроллера и мачты были взяты из китайского каталога цен, опубликованного компанией
\textit{Zonhan}. В связи с этим в дальнейших расчётах для данных компонентов
учитывались дополнительные затраты на международную транспортировку и налог
на добавленную стоимость.

\pagebreak
Ниже приведён перечень дополнительных статей затрат, включённых в итоговую
стоимость ветроэнергетической установки.

\begin{itemize}
    \item стоимость аккумуляторной батареи~--- $25\,000$ руб.
    за $1~\text{кВт}\cdot\text{ч}$ ёмкости. 
    Вместе с предыдущими компонентами формируют цену оборудования ($C_{\text{оборудования}}$)
    \item стоимость всех разрешений и проекта ~--- $50000$ руб.
    \item транспортные расходы из китая~--- $0.15 \cdot C_{\text{оборудования из Китая}}$ руб.
    \item локальная транспортировка~--- $0.08 \cdot C_{\text{оборудования из России}}$ руб.
    \item монтажные работы~--- $0.08 \cdot C_{\text{оборудования}}$ руб.
    \item сервисное обслуживание ~-- $0.15 \cdot C_{\text{оборудования}}$ руб.
    \item непредвиденные расходы --- $0.10 \cdot C_{\text{оборудования}}$ руб.
\end{itemize}

Посмотрим на список цен для ветрогенератора на 30 кВт, мачтой 20 метров и батареей на 15 кВт.

\begin{table}[h]
\centering
\caption{Оценка стоимости разных компонентов ВЭУ}
\begin{tabular}{ll}
\toprule
Статья затрат & Стоимость, руб \\
\midrule
Генератор & 1,416,420 \\
Контроллер & 267,615 \\
Мачта & 1,099,861 \\
Инвертор & 672,299 \\
Аккумуляторы & 375,000 \\
Доставка из Китая & 417,584 \\
Локальная транспортировка & 83,784 \\
Монтаж & 306,496 \\
Непредвиденные расходы & 383,120 \\
Обслуживание & 574,679 \\
Разрешения / проект & 50,000 \\
\midrule
Цена за кВт, руб/кВт & 186,562 \\
ИТОГО, руб & 5,596,859 \\
\bottomrule
\end{tabular}
\end{table}
\pagebreak
\subsection{Расчет времени окупаемости}

Теперь у нас есть все необходимые данные для оценки срока окупаемости
ветроэнергетической установки. Для этого воспользуемся следующей формулой:

\begin{equation}
T =
\frac{24 \cdot C_{\text{сумма}}}
{P_{\text{потребление}} \cdot C_{\text{покупка}} + P_{\text{избыток}} \cdot C_{\text{продажа}}}
\end{equation}

где:

\begin{itemize}
\item $C_{\text{сумма}}$~--- суммарная стоимость ветроэнергетической установки
\item $P_{\text{потребление}}$~--- объём потреблённой электроэнергии за период
2001--2024~гг.
\item $P_{\text{избыток}}$~--- объём избыточной электроэнергии, переданной в сеть
за период 2001--2024~гг. 
\item $C_{\text{покупка}}$~--- $6~\text{руб.}/\text{кВт}\cdot\text{ч}$
\item $C_{\text{продажа}}$~--- $2.5~\text{руб.}/\text{кВт}\cdot\text{ч}$

\end{itemize}

Теперь приведем результаты расчета.

\begin{figure}[!htbp] 
    \centering
    \includegraphics[width=0.9\textwidth, draft=false]{figures/6_Окупаемость при высоте башни 10 м в городе Казань.png}
    \caption{Окупаемость при высоте башни 10 м в городе Казань}
\end{figure}

\pagebreak
\begin{figure}[!htbp] 
    \centering
    \includegraphics[width=0.9\textwidth, draft=false]{figures/6_Окупаемость при высоте башни 10 м в городе Набережные Челны.png}
    \caption{Окупаемость при высоте башни 10 м в городе Набережные Челны}
\end{figure}

\begin{figure}[!htbp] 
    \centering
    \includegraphics[width=0.9\textwidth, draft=false]{figures/6_Окупаемость при высоте башни 10 м в городе Таганрог.png}
    \caption{Окупаемость при высоте башни 10 м в городе Таганрог}
\end{figure}

\pagebreak
\begin{figure}[!htbp] 
    \centering
    \includegraphics[width=0.9\textwidth, draft=false]{figures/6_Окупаемость при высоте башни 20 м в городе Казань.png}
    \caption{Окупаемость при высоте башни 20 м в городе Казань}
\end{figure}


\begin{figure}[!htbp] 
    \centering
    \includegraphics[width=0.9\textwidth, draft=false]{figures/6_Окупаемость при высоте башни 20 м в городе Набережные Челны.png}
    \caption{Окупаемость при высоте башни 20 м в городе Набережные Челны}
\end{figure}

\pagebreak
\begin{figure}[!htbp] 
    \centering
    \includegraphics[width=0.9\textwidth, draft=false]{figures/6_Окупаемость при высоте башни 20 м в городе Таганрог.png}
    \caption{Окупаемость при высоте башни 20 м в городе Таганрог}
\end{figure}

Видно, что во всех трёх городах установка ветроэнергетической установки (ВЭУ) является экономически обоснованной. 
Так, в Казани можно установить ВЭУ мощностью 12~кВт, которая окупится примерно за 20 лет.
